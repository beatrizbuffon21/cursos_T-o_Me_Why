\documentclass{article}
\usepackage[utf8]{inputenc}

\title{Aula 01 - Um pouco sobre a área de dados}
\author{Beatriz Woos Buffon}
\date{2025}

\begin{document}

\maketitle

\section*{Coleta de dados}

\begin{itemize}
    \item \textbf{Localização}: Exemplo: ao abrir o Google Maps, aparecem informações como tempo de rota, avaliações do lugar, quilometragem e outras métricas coletadas.
    \item \textbf{Redes sociais (Instagram)}: Dados sobre o que você comenta, curte, quantos Reels assiste e por quanto tempo permanece no aplicativo.
    \item \textbf{Navegadores}: Informações sobre quanto tempo você passa em cada site, sua jornada de compra, forma de pagamento utilizada etc.
    \item \textbf{Farmácias}: Geralmente o primeiro dado solicitado é o CPF. Tanto o varejo quanto a indústria farmacêutica coletam e analisam essas informações para entender hábitos de consumo.
    \item \textbf{Área financeira}: Foi uma das primeiras a aplicar modelos de dados, especialmente no \textit{score de crédito}. Utiliza dados para prever capacidade de pagamento, risco de inadimplência, concessão de financiamento e prevenção a fraudes.
    \item \textbf{Serviços de streaming (Netflix e similares)}: Recomendação personalizada de filmes e séries com base no que você assiste, curte ou adiciona à lista.
    \item \textbf{Varejo}: Uso de dados para identificar \textit{clusters} de clientes, definir estratégias de marketing e prever a possibilidade de um cliente deixar de consumir a marca.
\end{itemize}

\section*{Fontes de coleta de dados estruturada}
\begin{itemize}
    \item Pesquisas e censos (IBGE, Kantar, Datafolha, Nielsen, Ipsos).
    \item Bases governamentais e institucionais.
    \item Registros administrativos (cadastros de clientes, notas fiscais).
    \item Dados transacionais e digitais (e-commerce, aplicativos, sensores).
\end{itemize}

\section*{Possibilidades em Dados}
\begin{itemize}
    \item Previsão de vendas
    \item Tempo de garantia
    \item Limite de crédito
    \item Manutenção de equipamentos industriais
    \item Recomendação de produtos
    \item Definição de metas
    \item Processamento de Linguagem Natural (PLN)
    \item Conhecer diferentes grupos de clientes
    \item Validar produtos com usuários
    \item Entender sazonalidade
    \item Visão computacional
    \item Avaliar performance de atletas
    \item Medir eficácia de vacinas e medicamentos
    \item Apoiar na tomada de melhores decisões
\end{itemize}

\end{document}
